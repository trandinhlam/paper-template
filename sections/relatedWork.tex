\section{Related Work}
Richardson et al. \cite{Baroni1984} first started the study of influence maximization problem using proba-
bilistic approaches and later Kempe et al. formulated the problem as finding a small subset
of nodes k that maximizes the expected number of influenced nodes under a stochastic cas-
cade model. This problem is proved to be NP-hard and a greedy algorithm is provided to
solve the problem. However, this algorithm has a huge drawback of efficiency, so a lot of
more recent work have been focusing on improving the scalability of the algorithm. In \cite{Baroni1985},
2Leskovec et al. proposed ”lazy forward” algorithm which largely improved the efficiency of
the algorithm by exploiting the submodularity property of the objective matrix. Chen et al.
\cite{Baroni1986} later proposed yet another greedy algorithm for with new degree discount heuristics that
further improves the efficiency even further while achieving a matching performance to the
original greedy algorithm \cite{Baroni1987}.